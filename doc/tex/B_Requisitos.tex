\apendice{Especificación de Requisitos}

\section{Introducción}

En esta sección se describirán las funciones para las que la herramienta fue desarrollada. Se hará una detallada definición tanto de objetivos como de sus requisitos (funcionales y no funcionales) y también se hará un estudio en profundidad de los casos de uso.

\section{Objetivos generales}

Como ya se describió previamente en a memoria, hay un conjunto de objetivos en torno a los cuales se estructura el desarrollo del proyecto. A continuación se volverán a resumir los relacionados con la funcionalidad y las técnicas empleadas:

\begin{itemize}

\item El principal objetivo es crear una aplicación web en la que el usuario pueda almacenar noticias tanto por medio de web scraping como lectura de PDF.

\item Se busca que el usuario pueda añadir etiquetas  relacionadas con diferentes temas de debate a estas noticias, tanto manualmente como guardando una predicción realizada con aprendizaje automático.

\item Otro objetivo es que el usuario pueda añadir temas de debate nuevos que contengan dos clases, para luego poder etiquetar estas noticias en función de estas nuevas clases.

\item En caso de que el usuario ya tuviera un corpus preparado antes de comenzar a usar la aplicación, tiene la posibilidad de añadir un corpus mediante un archivo XML con todas las noticias que lo forman y sus etiquetas correspondientes.

\item La aplicación debe ser responsive en los diferentes equipos en los que se use, y fácil de usar para nuevos usuarios.


\end{itemize}

\section{Catalogo de requisitos}

\subsection{Requisitos Funcionales}

Estos son los requisitos funcionales que la herramienta debe cumplir:

\begin{itemize}

\item RF - 1: Los usuarios deben poder visualizar un conjunto de noticias a partir de la búsqueda por palabra clave.

\item RF - 2: Los usuarios podrán etiquetar manualmente las noticias en función del tema de debate seleccionado.
\begin{itemize}
\item RF - 2.1: Los usuarios podrán elegir de una lista de opciones el tema de debate mostrado en ese momento.
\end{itemize}

\item RF - 3: Los usuarios pueden ver una explicación gráfica para la predicción de la etiqueta de una noticia.
\begin{itemize}
\item RF - 3.1: Los usuarios también podrán guardar esa predicción como la etiqueta final de la noticia.
\end{itemize}

\item RF - 4: Los usuarios podrán guardar las noticias guardadas en archivos PDF almacenados en una carpeta concreta del proyecto.

\item RF - 5: Los usuarios podrán guardar nuevas noticias de 
los RSS de varios medios de comunicación españoles.
\begin{itemize}
\item RF - 5.1: Los usuarios podrán elegir de una lista de opciones el medio de comunicación del que ver las nuevas noticias.
\end{itemize}

\item RF - 6: Los usuarios podrán añadir un corpus de noticias entero, indicando el tema de debate y la etiqueta de cada noticia, a través de un archivo XML.

\item RF - 7: Los usuarios podrán crear nuevos temas de debate, escribiendo el nombre del propio tema y el nombre de dos clases, que se usarán posteriormente de etiquetas para ese tema.

\end{itemize}

\subsection{Requisitos No Funcionales}

Por otro lado, estos son los requisitos no relacionados con la funcionalidad que se deben tener en cuenta:

\begin{itemize}

\item RNF - 1: La interfaz de las pestañas no debe impedir su correcto funcionamiento en pantallas de diferentes tamaños.

\item RNF - 2: A pesar de que se produzcan picos de carga de trabajo, el funcionamiento general no debe producir tiempos de carga excesivos.

\item RNF - 3: La herramienta debe poder ejecutarse en cualquier navegador (siempre que soporten librerías como Bootstrap o JQuery).

\end{itemize}

\section{Especificación de requisitos}

A continuación, se describirán con detalle los casos de uso de los que se compone el proyecto.

\subsection{Diagrama de casos de uso}

\imagen{casosdeuso}{Diagrama de casos de uso}

\subsection{Descripción de casos de uso}

\subsubsection{CU1 - Consulta de Noticias}

\begin{center}
\begin{tabular}{ | m{3cm} | m{10cm}| } 
\hline
Caso de uso & Consulta de Noticias \\ 
\hline
Autor & Miguel Rodríguez \\ 
\hline
Requisitos & RF - 1: Los usuarios deben poder visualizar un conjunto de noticias a partir de la búsqueda por palabra clave \\ 
\hline
Descripción & El usuario introduce una palabra clave para la búsqueda de las noticias que la contengan \\
\hline
Precondiciones & Las noticias tienen que estar almacenadas en la base de datos. 

La palabra debe aparecer en la noticia para que sea devuelta. \\
\hline
Acciones & En la pestaña inicial, el usuario introduce en el cuadro de texto la palabra clave con la que quiere realizar la búsqueda y pulsa el botón ``Buscar'' \\
\hline
Postcondiciones & Se cargará una tabla con las noticias que contenían la palabra \\
\hline
Excepciones & En caso de que la palabra no se encuentre en ninguna noticia, no se devolverán filas de la tabla de resultados \\
\hline 
Importancia & Alta \\
\hline 
\end{tabular}
\end{center}

\subsubsection{CU2 - Etiquetado Manual}

\begin{center}
\begin{tabular}{ | m{3cm} | m{10cm}| } 
\hline
Caso de uso & Etiquetado Manual \\ 
\hline
Autor & Miguel Rodríguez \\ 
\hline
Requisitos & RF - 2: Los usuarios pueden etiquetar manualmente las noticias en función del tema de debate seleccionado

RF - 2.1: Los usuarios podrán elegir de una lista de opciones el tema de debate mostrado en ese momento \\ 
\hline
Descripción & El usuario selecciona un tema de debate y una de las dos clases para la noticia que quiera etiquetar \\
\hline
Precondiciones & La noticia que se va a etiquetar tiene que estar en la tabla de resultados. El tema de debate sobre el que se quiera etiquetar tiene que existir previamente. \\
\hline
Acciones & El usuario selecciona de una lista desplegable el tema de debate en el que quiera etiquetar las noticias. Una vez seleccionado, pulsando el botón de ``Etiquetar'' de cada fila podrá marcar cuál de las dos etiquetas quiere aplicar a esa noticia. Pulsando en ``Guardar Cambios'', se actualizarán todas las etiquetas modificadas. \\
\hline
Postcondiciones & Se añadirá o etiquetará esa etiqueta a esa noticia. Esa noticia pasa a formar parte del corpus de entrenamiento para ese tema de debate. \\
\hline
Excepciones & Si se realizan etiquetados pero no se guardan los resultados, no se aplicarán los cambios \\
\hline 
Importancia & Alta \\
\hline 
\end{tabular}
\end{center}

\subsubsection{CU3 - Creación de Explicación}

\begin{center}
\begin{tabular}{ | m{3cm} | m{10cm}| } 
\hline
Caso de uso & Creación de explicación \\ 
\hline
Autor & Miguel Rodríguez \\ 
\hline
Requisitos & RF - 3: Los usuarios pueden ver una explicación gráfica para la predicción de la etiqueta de una noticia.

RF - 3.1: Los usuarios también podrán guardar esa predicción como la etiqueta final de la noticia.\\ 
\hline
Descripción & El usuario selecciona la opción de explicación para una noticia de la tabla de resultados. \\
\hline
Precondiciones & La noticia que se va a etiquetar tiene que estar en la tabla de resultados. Debe existir un corpus de entrenamiento para el tema de debate sobre el que se va a realizar la predicción. \\
\hline
Acciones & El usuario selecciona de una lista desplegable el tema de debate en el que quiera etiquetar las noticias. Una vez seleccionado, pulsando el botón de ``Clasificar'' de una noticia en concreto cargará una página con la explicación correspondiente. Pulsando en ``Guardar Predicción'', se guardará el resultado de la predicción como etiqueta. \\
\hline
Postcondiciones & Se etiquetará la noticia con el valor de la clase ganadora en la predicción \\
\hline
Excepciones & Si no hay un corpus de entrenamiento, la explicación no se cargará correctamente \\
\hline 
Importancia & Alta \\
\hline 
\end{tabular}
\end{center}

\subsubsection{CU4 - Lectura de PDF}

\begin{center}
\begin{tabular}{ | m{3cm} | m{10cm}| } 
\hline
Caso de uso & Lectura de PDF \\ 
\hline
Autor & Miguel Rodríguez \\ 
\hline
Requisitos & RF - 4: Los usuarios podrán guardar las noticias guardadas en archivos PDF almacenados en una carpeta concreta del proyecto\\ 
\hline
Descripción & Se muestra una lista de noticias leídas en archivos PDF y el usuario escoge si guardar las nuevas o no. \\
\hline
Precondiciones & Debe haber archivos PDF que contengan noticias en el directorio adecuado. 

Además, deben de seguir el formato especificado para que se puedan leer correctamente. \\
\hline
Acciones & En la pestaña de lectura de PDF, el usuario pulsará el botón de leer noticias. 

Cuando se hayan leído y se muestre la lista, el usuario puede pulsar el botón de ``Guardar Nuevas'' para almacenar las noticias no previamente guardadas. \\
\hline
Postcondiciones & Se almacenarán las noticias no guardadas previamente en la base de datos. \\
\hline
Excepciones & Si no hay archivos PDF no se mostrará nada en la tabla de resultados. 

Si no tienen el formato indicado, no se podrá leer esas noticias. \\
\hline 
Importancia & Media \\
\hline 
\end{tabular}
\end{center}

\subsubsection{CU5 - Lectura de RSS}

\begin{center}
\begin{tabular}{ | m{3cm} | m{10cm}| } 
\hline
Caso de uso & Lectura de RSS \\ 
\hline
Autor & Miguel Rodríguez \\ 
\hline
Requisitos & RF - 5: Los usuarios podrán guardar nuevas noticias de 
los RSS de varios medios de comunicación españoles.

RF - 5.1: Los usuarios podrán elegir de una lista de opciones el medio de comunicación del que ver las nuevas noticias.
\\ 
\hline
Descripción & El usuario elige un medio de comunicación y después puede guardar las nuevas noticias de ese medio. \\
\hline
Precondiciones & El medio de comunicación tiene que ser uno de los incluidos en la lista. Hay que tener conexión a internet para acceder a los RSS de estos medios. \\
\hline
Acciones & En la pestaña de lectura de RSS, el usuario elegirá en un desplegable el medio de comunicación del que quiere leer la noticia.

Pulsando el botón de ``Guardar Nuevas'', el usuario podrá almacenar las que no estuvieran ya en la base de datos. \\
\hline
Postcondiciones & Se mostrará una tabla con las últimas noticias del medio de comunicación seleccionado. \\
\hline
Excepciones & En caso de que no haya conexión a Internet, la lectura de noticias no se podrá realizar. \\
\hline 
Importancia & Alta \\
\hline 
\end{tabular}
\end{center}

\subsubsection{CU6 - Lectura de XML}

\begin{center}
\begin{tabular}{ | m{3cm} | m{10cm}| } 
\hline
Caso de uso & Lectura de XML \\ 
\hline
Autor & Miguel Rodríguez \\ 
\hline
Requisitos & RF - 6: Los usuarios podrán añadir un corpus de noticias entero, indicando el tema de debate y la etiqueta de cada noticia, a través de un archivo XML. \\ 
\hline
Descripción & El usuario ordena leer los documentos XML que contienen los corpus situados en el directorio especificado.\\
\hline
Precondiciones & Los documentos XML tienen que tener el formato determinado. \\
\hline
Acciones & En la pestaña de lectura de nuevos conjuntos, el usuario pulsará el botón de ``Leer''. 

Se cargará una nueva pestaña donde si el usuario pulsa ``Guardar Nuevas'', se almacenarán las noticias, con las etiquetas ya aplicadas de antemano. \\
\hline
Postcondiciones & Se guardarán las nuevas noticias con etiquetas ya aplicadas, al contrario que en los otros formatos. \\
\hline
Excepciones & En caso de que no hubiera archivos XML o que no tengan el formato correcto, la lectura no se realizará correctamente. \\
\hline 
Importancia & Alta \\
\hline 
\end{tabular}
\end{center}

\subsubsection{CU7 - Creación de Dataset}

\begin{center}
\begin{tabular}{ | m{3cm} | m{10cm}| } 
\hline
Caso de uso & Creación de dataset \\ 
\hline
Autor & Miguel Rodríguez \\ 
\hline
Requisitos & RF - 7: Los usuarios podrán crear nuevos temas de debate, escribiendo el nombre del propio tema y el nombre de dos clases, que se usarán posteriormente de etiquetas para ese tema. \\ 
\hline
Descripción & El usuario tiene la opción de introducir nuevos dataset para las noticias indicando su nombre y el de sus 2 posibles clases. \\
\hline
Precondiciones & Los temas de debate deben de tener 2 clases posibles. \\
\hline
Acciones & En la pestaña inicial, el usuario debe pulsar el botón de ``Añadir Nuevo Dataset''.

En la pestaña a la que ha sido redirigido, escribirá el nombre del nuevo dataset y el de sus dos clases, que serán las etiquetas que luego se podrán añadir a las noticias.

Al pulsar el botón de guardar cambios, se actualizará la base de datos con esta información. \\
\hline
Postcondiciones & Se añadirá a la lista de datasets el escrito en la aplicación, y a todas las noticias almacenadas también se les añadirán campos para soportar las nuevas etiquetas.\\
\hline
Excepciones & En caso de que no se rellenen los tres campos, no se guardará correctamente el nuevo dataset. \\
\hline 
Importancia & Alta \\
\hline 
\end{tabular}
\end{center}