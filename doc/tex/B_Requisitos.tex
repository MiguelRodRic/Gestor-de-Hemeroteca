\apendice{Especificación de Requisitos}

\section{Introducción}

En esta sección se describirán las funciones para las que la herramienta fue desarrollada. Se hará una detallada definición tanto de objetivos como de sus requisitos (funcionales y no funcionales) y también se hará un estudio en profundidad de los casos de uso.

\section{Objetivos generales}

Como ya se describió previamente en a memoria, hay un conjunto de objetivos en torno a los cuales se estructura el desarrollo del proyecto. A continuación se volverán a resumir los relacionados con la funcionalidad y las técnicas empleadas:

\begin{itemize}

\item El principal objetivo es crear una aplicación web en la que el usuario pueda almacenar noticias tanto por medio de web scraping como lectura de PDF.

\item Se busca que el usuario pueda añadir etiquetas  relacionadas con diferentes temas de debate a estas noticias, tanto manualmente como guardando una predicción realizada con aprendizaje automático.

\item Otro objetivo es que el usuario pueda añadir temas de debate nuevos que contengan dos clases, para luego poder etiquetar estas noticias en función de estas nuevas clases.

\item En caso de que el usuario ya tuviera un corpus preparado antes de comenzar a usar la aplicación, tiene la posibilidad de añadir un corpus mediante un archivo XML con todas las noticias que lo forman y sus etiquetas correspondientes.

\item La aplicación debe ser responsive en los diferentes equipos en los que se use, y fácil de usar para nuevos usuarios.


\end{itemize}

\section{Catalogo de requisitos}

\subsection{Requisitos Funcionales}

Estos son los requisitos funcionales que la herramienta debe cumplir:

\begin{itemize}

\item RF - 1: Los usuarios deben poder visualizar un conjunto de noticias a partir de la búsqueda por palabra clave.

\item RF - 2: Los usuarios podrán etiquetar manualmente las noticias en función del tema de debate seleccionado.
\begin{itemize}
\item RF - 2.1: Los usuarios podrán elegir de una lista de opciones el tema de debate mostrado en ese momento.
\end{itemize}

\item RF - 3: Los usuarios pueden ver una explicación gráfica de la explicación para la predicción de la etiqueta de una noticia.
\begin{itemize}
\item RF - 3.1: Los usuarios también podrán guardar esa predicción como la etiqueta final de la noticia.
\end{itemize}

\item RF - 4: Los usuarios podrán guardar las noticias que se contuvieran en archivos PDF almacenados en una carpeta concreta del proyecto.

\item RF - 5: Los usuarios podrán guardar nuevas noticias de 
los RSS de varios medios de comunicación españoles.
\begin{itemize}
\item RF - 5.1: Los usuarios podrán elegir de una lista de opciones el medio de comunicación del que ver las nuevas noticias.
\end{itemize}

\item RF - 6: Los usuarios podrán añadir un corpus de noticias entero, indicando el tema de debate y la etiqueta de cada noticia, a través de un archivo XML.

\item RF - 7: Los usuarios podrán crear nuevos temas de debate, escribiendo el nombre del propio tema y el nombre de dos clases, que se usarán posteriormente de etiquetas para ese tema.

\end{itemize}

\subsection{Requisitos No Funcionales}

Por otro lado, estos son los requisitos no relacionados con la funcionalidad que se deben tener en cuenta:

\begin{itemize}

\item RNF - 1: La interfaz de las pestañas no debe impedir su correcto funcionamiento en pantallas de diferentes tamaños.

\item RNF - 2: A pesar de que se produzcan picos de carga de trabajo, el funcionamiento general no debe producir tiempos de carga excesivos.

\item RNF - 3: La herramienta debe poder ejecutarse en cualquier navegador (siempre que soporten librerías como Bootstrap o JQuery).

\end{itemize}

\section{Especificación de requisitos}

A continuación, se describirán con detalle los casos de uso de los que se compone el proyecto.

\subsection{Diagrama de casos de uso}

\imagen{casosdeuso}{Diagrama de casos de uso}

\subsection{Descripción de casos de uso}