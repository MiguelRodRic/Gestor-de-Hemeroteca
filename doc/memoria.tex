\documentclass[a4paper,11pt,twoside]{memoir}

% Citas indentadas
\usepackage{csquotes}

\usepackage{minted}


% Castellano
\usepackage[spanish,es-tabla]{babel}
\selectlanguage{spanish}
\usepackage[utf8]{inputenc}
\usepackage{placeins}

\RequirePackage{booktabs}
\RequirePackage[table]{xcolor}
\RequirePackage{xtab}
\RequirePackage{multirow}

% Links
\usepackage[colorlinks]{hyperref}
\hypersetup{
	allcolors = {red}
}

% Ecuaciones
\usepackage{amsmath}

% Rutas de fichero / paquete
\newcommand{\ruta}[1]{{\sffamily #1}}

% Párrafos
\nonzeroparskip


% Imagenes
\usepackage{graphicx}
\newcommand{\imagen}[2]{
	\begin{figure}[!h]
		\centering
		\includegraphics[width=0.9\textwidth]{#1}
		\caption{#2}\label{fig:#1}
	\end{figure}
	\FloatBarrier
}

\newcommand{\imagenalternativa}[3]{
	\begin{figure}[!h]
		\centering
		\includegraphics[width=0.9\textwidth]{#1}
		\caption[#3]{#2}\label{fig:#1}
	\end{figure}
	\FloatBarrier
}


\newcommand{\imagenflotante}[2]{
	\begin{figure}%[!h]
		\centering
		\includegraphics[width=0.9\textwidth]{#1}
		\caption{#2}\label{fig:#1}
	\end{figure}
}



% El comando \figura nos permite insertar figuras comodamente, y utilizando
% siempre el mismo formato. Los parametros son:
% 1 -> Porcentaje del ancho de página que ocupará la figura (de 0 a 1)
% 2 --> Fichero de la imagen
% 3 --> Texto a pie de imagen
% 4 --> Etiqueta (label) para referencias
% 5 --> Opciones que queramos pasarle al \includegraphics
% 6 --> Opciones de posicionamiento a pasarle a \begin{figure}
\newcommand{\figuraConPosicion}[6]{%
  \setlength{\anchoFloat}{#1\textwidth}%
  \addtolength{\anchoFloat}{-4\fboxsep}%
  \setlength{\anchoFigura}{\anchoFloat}%
  \begin{figure}[#6]
    \begin{center}%
      \Ovalbox{%
        \begin{minipage}{\anchoFloat}%
          \begin{center}%
            \includegraphics[width=\anchoFigura,#5]{#2}%
            \caption{#3}%
            \label{#4}%
          \end{center}%
        \end{minipage}
      }%
    \end{center}%
  \end{figure}%
}

%
% Comando para incluir imágenes en formato apaisado (sin marco).
\newcommand{\figuraApaisadaSinMarco}[5]{%
  \begin{figure}%
    \begin{center}%
    \includegraphics[angle=90,height=#1\textheight,#5]{#2}%
    \caption{#3}%
    \label{#4}%
    \end{center}%
  \end{figure}%
}
% Para las tablas
\newcommand{\otoprule}{\midrule [\heavyrulewidth]}
%
% Nuevo comando para tablas pequeñas (menos de una página).
\newcommand{\tablaSmall}[5]{%
 \begin{table}
  \begin{center}
   \rowcolors {2}{gray!35}{}
   \begin{tabular}{#2}
    \toprule
    #4
    \otoprule
    #5
    \bottomrule
   \end{tabular}
   \caption{#1}
   \label{tabla:#3}
  \end{center}
 \end{table}
}

%
% Nuevo comando para tablas pequeñas (menos de una página).
\newcommand{\tablaSmallSinColores}[5]{%
 \begin{table}[H]
  \begin{center}
   \begin{tabular}{#2}
    \toprule
    #4
    \otoprule
    #5
    \bottomrule
   \end{tabular}
   \caption{#1}
   \label{tabla:#3}
  \end{center}
 \end{table}
}

\newcommand{\tablaApaisadaSmall}[5]{%
\begin{landscape}
  \begin{table}
   \begin{center}
    \rowcolors {2}{gray!35}{}
    \begin{tabular}{#2}
     \toprule
     #4
     \otoprule
     #5
     \bottomrule
    \end{tabular}
    \caption{#1}
    \label{tabla:#3}
   \end{center}
  \end{table}
\end{landscape}
}

%
% Nuevo comando para tablas grandes con cabecera y filas alternas coloreadas en gris.
\newcommand{\tabla}[6]{%
  \begin{center}
    \tablefirsthead{
      \toprule
      #5
      \otoprule
    }
    \tablehead{
      \multicolumn{#3}{l}{\small\sl continúa desde la página anterior}\\
      \toprule
      #5
      \otoprule
    }
    \tabletail{
      \hline
      \multicolumn{#3}{r}{\small\sl continúa en la página siguiente}\\
    }
    \tablelasttail{
      \hline
    }
    \bottomcaption{#1}
    \rowcolors {2}{gray!35}{}
    \begin{xtabular}{#2}
      #6
      \bottomrule
    \end{xtabular}
    \label{tabla:#4}
  \end{center}
}

%
% Nuevo comando para tablas grandes con cabecera.
\newcommand{\tablaSinColores}[6]{%
  \begin{center}
    \tablefirsthead{
      \toprule
      #5
      \otoprule
    }
    \tablehead{
      \multicolumn{#3}{l}{\small\sl continúa desde la página anterior}\\
      \toprule
      #5
      \otoprule
    }
    \tabletail{
      \hline
      \multicolumn{#3}{r}{\small\sl continúa en la página siguiente}\\
    }
    \tablelasttail{
      \hline
    }
    \bottomcaption{#1}
    \begin{xtabular}{#2}
      #6
      \bottomrule
    \end{xtabular}
    \label{tabla:#4}
  \end{center}
}

%
% Nuevo comando para tablas grandes sin cabecera.
\newcommand{\tablaSinCabecera}[5]{%
  \begin{center}
    \tablefirsthead{
      \toprule
    }
    \tablehead{
      \multicolumn{#3}{l}{\small\sl continúa desde la página anterior}\\
      \hline
    }
    \tabletail{
      \hline
      \multicolumn{#3}{r}{\small\sl continúa en la página siguiente}\\
    }
    \tablelasttail{
      \hline
    }
    \bottomcaption{#1}
  \begin{xtabular}{#2}
    #5
   \bottomrule
  \end{xtabular}
  \label{tabla:#4}
  \end{center}
}



\definecolor{cgoLight}{HTML}{EEEEEE}
\definecolor{cgoExtralight}{HTML}{FFFFFF}

%
% Nuevo comando para tablas grandes sin cabecera.
\newcommand{\tablaSinCabeceraConBandas}[5]{%
  \begin{center}
    \tablefirsthead{
      \toprule
    }
    \tablehead{
      \multicolumn{#3}{l}{\small\sl continúa desde la página anterior}\\
      \hline
    }
    \tabletail{
      \hline
      \multicolumn{#3}{r}{\small\sl continúa en la página siguiente}\\
    }
    \tablelasttail{
      \hline
    }
    \bottomcaption{#1}
    \rowcolors[]{1}{cgoExtralight}{cgoLight}

  \begin{xtabular}{#2}
    #5
   \bottomrule
  \end{xtabular}
  \label{tabla:#4}
  \end{center}
}


















\graphicspath{ {./img/} }

% Capítulos
\chapterstyle{bianchi}
\newcommand{\capitulo}[2]{
	\setcounter{chapter}{#1}
	\setcounter{section}{0}
	\chapter*{#2}
	\addcontentsline{toc}{chapter}{#2}
	\markboth{#2}{#2}
}

% Apéndices
\renewcommand{\appendixname}{Apéndice}
\renewcommand*\cftappendixname{\appendixname}

\newcommand{\apendice}[1]{
	%\renewcommand{\thechapter}{A}
	\chapter{#1}
}

\renewcommand*\cftappendixname{\appendixname\ }

% Formato de portada
\makeatletter
\usepackage{xcolor}
\newcommand{\tutor}[1]{\def\@tutor{#1}}
\newcommand{\course}[1]{\def\@course{#1}}
\definecolor{cpardoBox}{HTML}{E6E6FF}
\def\maketitle{
  \null
  \thispagestyle{empty}
  % Cabecera ----------------
\noindent\includegraphics[width=\textwidth]{cabecera}\vspace{1cm}%
  \vfill
  % Título proyecto y escudo informática ----------------
  \colorbox{cpardoBox}{%
    \begin{minipage}{.8\textwidth}
      \vspace{.5cm}\Large
      \begin{center}
      \textbf{TFG del Grado en Ingeniería Informática}\vspace{.6cm}\\
      \textbf{\LARGE\@title{}}
      \end{center}
      \vspace{.2cm}
    \end{minipage}

  }%
  \hfill\begin{minipage}{.20\textwidth}
    \includegraphics[width=\textwidth]{escudoInfor}
  \end{minipage}
  \vfill
  % Datos de alumno, curso y tutores ------------------
  \begin{center}%
  {%
    \noindent\LARGE
    Presentado por \@author{}\\ 
    en Universidad de Burgos --- \@date{}\\
    Tutores: \@tutor{}\\
  }%
  \end{center}%
  \null
  \cleardoublepage
  }
\makeatother

\newcommand{\nombre}{Miguel Rodríguez Rico} %%% cambio de comando

% Datos de portada
\title{Herramienta para el análisis y gestión de Hemeroteca}
\author{\nombre}
\tutor{Dr. José Francisco Díez Pastor,
	Dr. César I. García Osorio}
\date{\today}

\begin{document}

\maketitle


\newpage\null\thispagestyle{empty}\newpage


%%%%%%%%%%%%%%%%%%%%%%%%%%%%%%%%%%%%%%%%%%%%%%%%%%%%%%%%%%%%%%%%%%%%%%%%%%%%%%%%%%%%%%%%
\thispagestyle{empty}


\noindent\includegraphics[width=\textwidth]{cabecera}\vspace{1cm}

\noindent D. José Francisco Díez Pastor y César Ignacio García Osorio, profesores del departamento de Ingeniería Civil, área de Lenguajes y Sistemas Informáticos.

\noindent Exponen:

\noindent Que el alumno D. \nombre, con DNI 71286391F, ha realizado el Trabajo final de Grado en Ingeniería Informática titulado ``Herramienta para el análisis y gestión de Hemeroteca". 

\noindent Y que dicho trabajo ha sido realizado por el alumno bajo la dirección de los que suscriben, en virtud de lo cual se autoriza su presentación y defensa.

\begin{center} %\large
En Burgos, {\large \today}
\end{center}

\vfill\vfill\vfill

% Author and supervisor
\begin{minipage}{0.45\textwidth}
\begin{flushleft} %\large
Vº. Bº. del Tutor:\\[2cm]
Dr. José Francisco Díez Pastor
\end{flushleft}
\end{minipage}
\hfill
\begin{minipage}{0.45\textwidth}
\begin{flushleft} %\large
Vº. Bº. del Tutor:\\[2cm]
Dr. César Ignacio García Osorio
\end{flushleft}
\end{minipage}
\hfill

\vfill

% para casos con solo un tutor comentar lo anterior
% y descomentar lo siguiente
%Vº. Bº. del Tutor:\\[2cm]
%D. nombre tutor


\newpage\null\thispagestyle{empty}\newpage




\frontmatter

% Abstract en castellano
\renewcommand*\abstractname{Resumen}
\begin{abstract}
Se desea desarrollar una aplicación web en la que un usuario pueda almacenar un conjunto de noticias obtenidas de diferentes medios y formatos (páginas web, archivos PDF y XML) a modo de hemeroteca. Adicionalmente, el usuario podrá etiquetar estas noticias en relación a distintos temas y se usará la minería de datos para realizar predicciones sobre nuevas noticias.

Para realizar esta labor, previamente se ha llevado a cabo una investigación de diferentes tecnologías para elegir las que más se adecuan a los objetivos establecidos; tanto en el caso de la herramienta con la que se creará la aplicación web y la base de datos no relacional en la que se almacenan las noticias, como en lo respectivo al análisis de textos con el que se realizarán las predicciones. Las herramientas principales escogidas son Flask, MongoDB, Scikit-Learn y Lime.

Teniendo en cuenta que la herramienta está pensada para un uso ajeno al programador que la ha creado, al implementar
 la herramienta se ha dado importancia a la accesibilidad de la misma, de modo que los nuevos usuarios que tengan que usarla sin estar familiarizados puedan acometer un proceso de aprendizaje lo más ligero posible.

Por último, se ha implementado, haciendo uso de técnicas de machine learning, la posibilidad de que el usuario vea qué criterios ha seguido la herramienta de predicción a la hora de asignar una etiqueta a una noticia. La intención es que, inicialmente, el usuario introduzca un conjunto de noticias a las que ha dado etiqueta manualmente y, a partir de esa situación, comenzar a incluir nuevas noticias que recibirán su etiqueta de forma automática, basándose en las noticias que las precedan.


\end{abstract}

\renewcommand*\abstractname{Descriptores}
\begin{abstract}
Machine learning, Flask, Bases de Datos NoSQL, Python, Web Scraping, Aplicación web.
\end{abstract}

\clearpage

% Abstract en inglés
\renewcommand*\abstractname{Abstract}
\begin{abstract}
The aim of the project is to develop a web application where a user will be able to store a set of news obtained from different media and with different formats (whether it is a web page, a PDF file or a XML file) so that it can become a Newspaper Library. In addition, the user can label those news according to different aspects and data mining will be used in order to predict labels for incoming news.

To tackle this task, some research was previously carried out in order to choose the most suitable technologies to the established objectives, both in the case of the tools for creating the application and the NoSQL database, as well as the tools related to the text analysis that is going to undertake the predictions. The main chosen tools are Flask, MongoDB, Scikit-Learn and Lime.

Taking into consideration that the application is not going to be used by the programmer who coded it, the accessibility of the application itself was very important, so that the unprepared final user won't experiment a tough training process.

Finally, the possibility of the user watching the criteria used by the prediction tool when returning a label has been implemented using machine learning techniques. The aim is that the user starts off with storing a set of labeled news and, having done that, the news following those will be labeled by the prediction tool.

\end{abstract}

\renewcommand*\abstractname{Keywords}
\begin{abstract}
Machine learning, Flask, NoSQL Databases, Python, Web Scraping, Web application.
\end{abstract}

\clearpage

% Indices
\tableofcontents

\clearpage

\listoffigures

\clearpage

\listoftables
\clearpage

\mainmatter
\capitulo{1}{Introducción}

Los estudios sociológicos tienen un gran papel a la hora de comprender y describir comportamientos y fenómenos colectivos de la sociedad \cite{wiki:estudiosociologico}. Además de ésto, también sirven para que determinadas organizaciones den visibilidad a problemas que podrían no ser obvios para una gran parte de la población.

No obstante, las encuestas y métodos similares de recogida de datos no son las únicas formas de recabar información interesante respecto de la sociedad. Por ejemplo, realizando un análisis de otros campos como la prensa escrita, redes sociales o programas de televisión se puede encontrar patrones e intenciones donde se critican o defienden diferentes asuntos de actualidad de forma sistemática, especialmente cuando estas organizaciones tienen marcadas líneas ideológicas.

Un debate especialmente popular en los últimos meses es el tratamiento de la violencia de género por parte de los medios de comunicación, donde se pueden observan claras tendencias a denunciarlo o a enmascararlo, dependiendo de la fuente de las noticias \cite{RICD2653}.

Por este motivo, se decidió realizar este proyecto en colaboración con María Isabel Menéndez Menéndez, doctora e investigadora en la Universidad de Burgos, cuyo área de estudio se centra en el análisis de la comunicación desde la perspectiva de género. 

La intención es crear una aplicación en la que pueda crear una hemeroteca donde realizar estudios sociológicos a partir de patrones encontrados en las noticias almacenadas. Aunque inicialmente se limita a debates como el machismo y el vientre de alquiler, es posible crear en la herramienta más temas para análisis sociológico y reutilizar las noticias para estas nuevas perspectivas.


\include{./tex/2_Objetivos_del_proyecto}
\capitulo{3}{Conceptos teóricos}
A continuación se explicarán los conceptos de carácter teórico más relevantes del proyecto, de modo que se puedan comprender con más exactitud aquellas partes con una mayor complejidad.


\section{Minería de Datos}

En los siguientes apartados se van a desarrollar todos los conceptos del proyecto relacionados con la minería de datos, esenciales en la parte de predicción del perfil ideológico de las noticias.

Inicialmente, se resumirá de forma breve el concepto de aprendizaje automático para a continuación hablar de la técnica del bag of words y de los clasificadores que se han usado en combinación con ésta.

\subsection{Machine learning}

El aprendizaje automático, o machine learning \cite{andrieu2003introduction}, es una técnica del campo de la Inteligencia Artificial que permite a un computador simular un proceso de aprendizaje \cite{wiki:machinelearning}. Las predicciones que se realizan en esta aplicación se basan en el machine learning. En términos generales, se pueden distinguir tres tipos de aprendizaje distinto en función de la supervisión del programador:

\begin{itemize}

\item Aprendizaje supervisado, donde se indica a qué clase pertenece cada miembro del conjunto de entrenamiento.

\item Aprendizaje no supervisado, donde los elementos del conjunto de entrenamiento no tienen asignadas clases y a lo largo del proceso se crean de forma automática 
agrupaciones entre los elementos similares.

\item Aprendizaje semisupervisado, donde hay miembros del conjunto que tienen asignada una clase y otros no, para mejorar la exactitud del aprendizaje \cite{wiki:semisupervised}.

\end{itemize}

Respecto al proyecto, el aprendizaje que se ha realizado es de tipo supervisado, ya que en el corpus de entrenamiento todas las noticias tienen clase asignada.

No obstante, podría decirse que conceptualmente tiene elementos del aprendizaje semisupervisado, pues pasado cierto tiempo, lo más probable es que parte del corpus del usuario tenga la clase asignada por una predicción realizada previamente, y no por un etiquetado manual, algo que es característico de este tipo de aprendizaje.

\subsection{Bag of Words}

Bag of Words \cite{bagofwords} \cite{wiki:bagofwords} es una técnica de agrupación de palabras relacionada con el Procesamiento de Lenguaje Natural. Consiste en, a partir de un texto de muestra, almacenar la frecuencia de aparición de cada palabra distinta que esté contenida en el texto, independientemente del orden de aparición, y sin tener en cuenta el contexto en el que aparezca. 

En términos de estructuras de datos, un bag of words se suele almacenar con un formato de diccionario, donde la clave está formada por cada palabra distinta, y el valor
es la frecuencia de aparición de esa palabra.

Vamos a ilustrar lo explicado con un ejemplo. A continuación, mostramos unas frases de ejemplo:

\begin{itemize}

\item ``No se puede fumar en establecimientos públicos"

\item ``Se puede fumar en residencias privadas"

\item ``En general, no se debería fumar"

\end{itemize}

Teniendo estas frases, si almacenamos la frecuencia de las palabras distintas para cada una de ellas, y las combinamos, tendríamos una tabla como la 3.1.

\tablaSmall{Frecuencias de palabras en las frases de ejemplo}{c c}{ejemplofrecuenciaspalabras}
{ \multicolumn{1}{c}{Palabras} & Frecuencia \\}{ 
No & 2\\
Se & 3\\
Puede & 2\\
Fumar & 3\\
En & 3\\
Establecimientos & 1\\
Públicos & 1\\
Residencias & 1\\
Privadas & 1\\
General & 1\\
Debería & 1\\
} 

El potencial de esta técnica para análisis de texto es muy interesante. Por ejemplo, y por relacionarlo con este proyecto, se podría asociar una etiqueta a cada una de las frases propuestas, como ``A favor de los fumadores'' y ``En contra de los fumadores''; para que una vez realizado el bag of words se pueda, entre otras funcionalidades, deducir las probabilidades que tiene cada una de las palabras del vocabulario obtenido de pertenecer a una de las dos clases.

De forma indirecta, lo previamente mencionado se puede usar para predecir cuál sería la ideología de una frase nueva, en función de las palabras de la misma que coinciden con el vocabulario ya almacenado por los clasificadores.

En el apartado ``Conceptos Avanzados de Bag of Words'' dentro de ``Aspectos Relevantes'' se explicarán con más detenimiento algunos aspectos más profundos de esta técnica.

\subsection{Clasificador Naive Bayes}

Este clasificador es muy popular en la clasificación de textos. Basa su funcionamiento en el Teorema de Bayes, que consiste en calcular la probabilidad de un evento dando relevancia al conocimiento previo de los factores relacionados consigo mismo \cite{bayestheorem}.

Llevando esto a la minería de datos, las probabilidades de que un elemento pertenezca a una clase van a ser potenciadas por el conocimiento previo del valor de cada uno de sus atributos. Además, se le añade la etiqueta de ``naive'' porque se considera que no hay ninguna relación de dependencia entre los diferentes atributos que componen ese elemento, aunque fuera del contexto de la clasificación estos atributos sí que estén fuertemente relacionados \cite{bayesclassifier}.

\subsection{Árbol de decisión}

Un árbol de decisión es una técnica para la toma de decisiones basado en la evaluación de atributos por separado. Cada árbol tiene un número de hojas, correspondiente al número de clases que se pueden devolver como resultado en función de la decisión que se haya tomado en la ``raíz'' del mismo \cite{decisiontree}.

\imagen{decisiontree1}{Ejemplo básico de funcionamiento de árbol de decisión}

Cuanto más complejo sea el árbol y mayor el número de nodos, se podrán evaluar decisiones más complejas y tener varios atributos en cuenta. Es una técnica usada con mucha frecuencia en el aprendizaje automático, ya que la combinación de muchos árboles de decisión forman lo que se conocen como ``bosques'', de los que hablaremos a continuación.

\subsection{Bagging}

También conocido como Bootstrap Agreggation \cite{breiman1996bagging}, es una técnica de aprendizaje supervisado, y más concretamente de clasificación, basada en la combinación de clasificadores. Su objetivo es el de reducir todo lo posible la varianza en la clasificación de datos. Consiste en obtener una predicción por votación a partir del promedio de los resultados obtenidos aplicando un gran número de clasificadores a distintas partes de los datos de entrenamiento.

De forma breve, podríamos explicar el proceso que realiza el Bagging en los siguientes pasos:

\begin{enumerate}

\item  A partir del conjunto de entrenamiento, se obtienen $m$ subconjuntos distintos, llamados ``bags'', que contienen combinaciones aleatorias de instancias del conjunto de entrenamiento.

\item Para cada ``bag'', se entrena un modelo con un clasificador distinto.

\item Todos los modelos resultantes se usan posteriormente para predecir el conjunto de prueba.

\item Para saber el resultado final, basta con saber cuál fue la clase que obtuvo más predicciones de todas las que se realizaron.

\end{enumerate}

Ésta técnica se usa mucho en los casos de predicciones basadas en árboles, como es el caso de los bosques aleatorios, que explicaremos a continuación. En la siguiente imagen se pueden observar de forma sencilla los pasos previamente mencionados.

\imagen{Bagging}{Esquema de funcionamiento básico de Bagging  \cite{baggingpicture}}


\subsection{Clasificador Random Forest}

El clasificador Random Forest es una técnica derivada del bagging aplicada a árboles de decisión \cite{randomforestclassifier}. Como su nombre sugiere, se basa en el uso de la combinación de varios árboles de clasificación. 

Cada uno de estos árboles se encarga de entrenar una instancia de los datos de entrenamiento, y la solución devuelta será la clase obtenida por votación de todos los árboles. 

El principal objetivo de realizar este proceso es el de, mediante los subconjuntos aleatorios, aumentar todo lo posible la precisión y controlar el sobreajuste de los resultados. Por ese motivo, si tenemos un mismo conjunto de entrenamiento para todas las pruebas, se obtendrá un menor margen de error en aquellas clasificaciones que tengan un mayor número de árboles, mientras que los resultados con un número mínimo de árboles no serán tan fiables \cite{randomforestclassifierberkeley}.

En la siguiente imagen se muestra un esquema de funcionamiento del bosque aleatorio, en la que se puede observar las similaridades con el proceso de Bagging explicado anteriormente.

\imagen{RandomForestExample}{Esquema de funcionamiento de bosque aleatorio \cite{randomforestpicture}}


\subsection{Vectorizador de texto}

Herramientas que convierten un documento de texto en matrices de tokens, que usualmente son palabras  \cite{countvectorizer}.
Son muy útiles para aplicar la técnica del bag of words a cualquier texto sin tener que recurrir a un conteo manual.

El tamaño de la matriz suele corresponderse con el tamaño del vocabulario obtenido a partir de la extracción de palabras del texto, aunque en la mayoría de herramientas el propio programador puede elegir un límite fijo para el tamaño. 

También es muy frecuente encontrar proyectos en los que se han usado vectorizadores para el tratamiento de imágenes, los cuales permiten transformarlas en formatos más flexibles para que puedan ser manejadas y exportadas con facilidad. No obstante, en contextos de investigación y análisis de textos, estas herramientas también son especialmente prácticas.



\section{Web Scraping}

El Web Scraping \cite{mitchell2015web} es una técnica, principalmente alternativa al uso de APIs de los propietarios de páginas web, para extraer información relevante de las propias páginas web.

De forma más concreta, lo que realiza un web scraper es la detección de información que se encuentra dentro de etiquetas concretas de un documento de tipo HTML, y su posterior extracción al indicarle las etiquetas en las que buscar. Es una técnica con un gran potencial, pues le evita al usuario la búsqueda manual de esta información, lo cual es una tarea especialmente ardua en los complejos códigos HTML de la mayoría de páginas web. 

A continuación mostramos un ejemplo sencillo de Web Scraping escrito en Python usando urllib2 y BeautifulSoup, en el que obtenemos el número de descargas de un repositorio en SourceForge:

\begin{enumerate}

\item Hacemos una petición http con la url indicada:

\imagen{WebScraping1}{Petición http}

\item Leemos el contenido de la url y se la pasamos al scraper:

\imagen{WebScraping2}{Lectura del contenido de la web}

\item A través del scraper, accedemos al texto de la etiqueta que queramos directamente:

\imagen{WebScraping3}{Acceso a la información deseada con el scraper}

\end{enumerate}




\include{./tex/4_Tecnicas_y_herramientas}
\capitulo{5}{Aspectos relevantes del desarrollo del proyecto}

En este capítulo se desarrollarán en profundidad aspectos sólo mencionados en capítulos previos, relacionados tanto con conceptos teóricos como con las técnicas y herramientas, además de dar explicación a cuestiones relevantes respecto de la aplicación o de su desarrollo.

\section{Conceptos avanzados de Bag of Words}

Además de lo mencionado en los conceptos teóricos, hay diferentes matices dentro del bag of words que merecen la pena ser explicados con más detalle, ya que fueron relevantes en la implementación del proyecto.

\subsection{Stopwords}

Se conoce como \emph{``stopwords''} (o palabras vacías) al conjunto de palabras prefijadas por el programador (o por el usuario, si se le da la opción) que no queremos que se tengan en cuenta a la hora de formar un vocabulario con el texto de entrada. 

Cuando se está dando forma al modelo, se le pasa este conjunto a la herramienta que va a transformar el texto en un vocabulario con frecuencia de palabras, de modo que ésta filtra las palabras mencionadas para eliminar el ``ruido'' y de forma consecuente, aumentar la relevancia del resto de palabras que almacene.

Habitualmente, lo que se busca evitar son las palabras que tienen menos significado propio dentro de una frase. Dicho de otra manera, las primeras palabras que deben ser filtradas son artículos, determinantes, preposiciones... que son las menos interesantes en una frase.

En este proyecto, se hizo uso de un conjunto de \emph{stopwords} de palabras en español que viene incluido en la librería de NLTK \cite{nltk}, que contiene las palabras de la siguiente lista:

\begin{displayquote}
\textbf{  de, la, que, el, en, y, a, los, del, se, las, por, un, para, con, no, una, su, al, lo, como, más, pero, sus, le, ya, o, este, porque, esta, entre, sí, cuando, muy, sin, sobre,
 también, me, hasta, hay, donde, quien, desde, todo, nos, durante, todos, uno, les, ni, contra, otros, ese, eso, ante, ellos, e, esto, mí, antes, algunos, qué, unos, yo, otro, otras, otra, él, tanto, esa, estos, mucho, quienes, nada, muchos, cual, poco, ella, estar, estas, algunas, algo, nosotros, mi, mis, tú, te, ti, tu, tus, ellas, nosotras, vosostros, vosostras, os, mío, mía, míos, mías, tuyo, tuya, tuyos, tuyas, suyo, suya, suyos, suyas, nuestro, nuestra, nuestros, nuestras, vuestra, vuestros, vuestras, esos, esas, estoy, estás, está, estamos, estáis, están, esté, estés, estemos, estéis, estén, estaré, estarás, estará, estaremos, estaréis, estarán, estaría, estarías, estaríamos, estaríais, estarían, estaba, estabas, 
 estábamos, estabais, estaban, estuve, estuviste, estuvo, estuvimos, estuvisteis, estuvieron, estuviera, estuvieras, estuviéramos, estuvierais, estuvieran, estuviese, estuvieses, 
 estuviésemos, estuvieseis, estuviesen, estando, estado, estada, estados, estadas, estad, he, has, ha, hemos, habéis, han, haya, hayas, hayamos, hayáis, hayan, habré, habrás, habrá, habremos, habréis, habrán, habría, habrías, habríamos, habríais, habrían, había, habías, habíamos, habíais, habían, 
 hube, hubiste, hubo, hubimos, hubisteis, hubieron, hubiera, hubieras, hubiéramos, hubierais, hubieran, hubiese,
 hubieses, hubiésemos, hubieseis, hubiesen, habiendo, habido, habida, habidos, habidas, soy, eres, es, somos, sois, son, sea, 
 seas, seamos, seáis, sean, seré, serás, será, seremos, 
 seréis, serán, sería, serías, seríamos, seríais, serían, era, 
 eras, éramos, erais, eran, fui, fuiste, fue, fuimos, 
 fuisteis, fueron, fuera, fueras, fuéramos, fuerais, fueran, fuese, fueses, fuésemos, fueseis, fuesen, sintiendo, sentido, sentida, sentidos, sentidas, siente, sentid, tengo, 
 tienes, tiene, tenemos, tenéis, tienen, tenga, tengas, tengamos, tengáis, tengan, tendré, tendrás, tendrá, tendremos, tendréis, tendrán, tendría, tendrías, tendríamos, tendríais, 
 tendrían, tenía, tenías, teníamos, teníais, tenían, tuve, tuviste, tuvo, tuvimos, tuvisteis, tuvieron, tuviera, tuvieras, tuviéramos, tuvierais, tuvieran, tuviese, tuvieses, tuviésemos, 
 tuvieseis, tuviesen, teniendo, tenido, tenida, tenidos, tenidas
}
\end{displayquote}


Ya que estas palabras son muy frecuentes en la mayoría de frases en español, no sólo restaban importancia a las verdaderamente relevantes, si no que incluso podrían ser consideradas de las más relevantes simplemente por su frecuencia de aparición, lo cuál restaría mucha calidad a la predicción posterior. 

Cabe destacar que ésto no soluciona todo el problema, ya que en los textos de las noticias podrían aparecer alguna de estas palabras con mayúsculas, sin tildes, u otras palabras no relevantes que evitarían el filtro. No obstante, con este filtro ya se puede asegurar una cierta calidad en las predicciones de forma general.

\subsection{Normalización de las frecuencias}

Como se ha mencionado previamente, lo habitual es que una predicción de clases basada en un bag of words se realice basando la clasificación en la frecuencia de aparición de determinadas palabras en los textos de cada una de las clases.

Para poder medir esta frecuencia, no es suficiente con tener los vocabularios de cada clase con sus frecuencias, pues no es una medida lo suficientemente precisa. Lo que se debe realizar es una normalización de estas frecuencias, teniendo en cuenta los siguientes cálculos \cite{spamtutorial}

\begin{itemize}

\item Frecuencia normalizada de una palabra en una clase:

\[ \mathit{FrecuenciaClase1}[ \mathit{palabra}] = \frac{\mathit{apariciones}[ \mathit{palabra}]}{\mathit{numeroNoticias}[ \mathit{Clase1}]} \]

Se recomienda añadir una aparición por defecto, para evitar divisiones entre 0.

\item Ratio de pertenencia de una palabra a una clase, en el caso de que el problema sólo tenga 2 clases:

\[ \mathit{RatioClase1}[ \mathit{palabra}] = \frac{\mathit{FrecuenciaClase1}[ \mathit{palabra}]}{\mathit{FrecuenciaClase2}[ \mathit{palabra}]} \]

Si, siguiendo el ejemplo que mencionamos en los conceptos teóricos, tenemos que la palabra ``salud'' tiene un mayor ratio en la clase ``En contra de los fumadores" que en ``A favor de los fumadores", se puede anticipar que la próxima vez que se reciba una frase o texto con la palabra ``salud'', tendrá mayores posibilidades de pertenecer a la clase ``En contra de los fumadores''.

\end{itemize}

Realizando estos simples cálculos y ordenando las palabras en función de sus respectivos ratio, se puede deducir cuáles serán las palabras más influyentes a la hora de saber si una frase o un texto se corresponde con una clase u otra.

\section{Resultados de la explicación de la predicción}

Una de las funcionalidades que más peso tiene en el producto final es el de mostrar una explicación de la predicción realizada sobre una noticia, respecto de un Dataset determinado. 

Siguiendo la filosofía de ``no reinventar la rueda'', para parte de la programación de este apartado hemos usado la librería Lime, de la cual se habla en el apartado de ``Técnicas y Herramientas'', y que permite obtener una explicación de cualquier clasificación realizada, ya sea usando texto o imagen, y preferiblemente usando Scikit-learn como librería para practicar la minería de datos.

Se ha aprovechado la posibilidad que ofrece Lime de devolver esta explicación con un formato de bloque HTML, en el cuál a su vez se incluyen tres apartados distintos:

\begin{itemize}

\item Barras de progreso que indican las probabilidades de pertenencia a cada una de las clases.

\item Un gráfico de barras en el que se muestran por orden las palabras más influyentes a la hora de tomar la decisión, ya sea para una clase o para otra.

\item El texto de prueba entero que se ha recibido, resaltando en él las palabras que se han incluido en el gráfico del punto anterior, para que el usuario pueda observar estas palabras en su contexto real (antes de hacer el bag of words), y no simplemente en la lista de palabras influyentes.

\end{itemize}

\imagen{CapturaExplicacion}{Resultados HTML de la explicación}

En la imagen se observan de forma diferenciada los tres apartados, y se le ofrece al usuario la posibilidad de, después de haber estudiado con detenimiento los detalles de la predicción, decidir etiquetar esta noticia en concordancia con la predicción realizada.

Puede que al usar la aplicación de la sensación de que el estilo visual de estos resultados de la explicación no encaja completamente con el estilo visual del resto de las pestañas. Esto es debido a que se inyecta directamente el bloque HTML sin alterar que devuelve Lime en la pestaña antes de cargar la misma. Debido a limitaciones de tiempo, era más productivo aprovechar este código HTML ya escrito que solamente recibir de Lime los datos y dedicar tiempo extra a escribir el código HTML y las funciones de Javascript necesarias para preparar la misma vista de resultados pero con otros estilos diferentes. De nuevo, se ha decidido seguir el razonamiento de que ``no hay que reinventar la rueda''.

\section{Pruebas de clasificación}

Como ya se ha mencionado en el apartado de conceptos teóricos, hay más de un clasificador que se tuvo en cuenta en el aprendizaje automático de textos. A continuación haremos una comparativa en la que se verá cómo afectaban al rendimiento y precisión cada caso.

\subsection{Predicción de clase}

Cuando se muestra una muestra de resultados de noticias, se puede observar una barra que indica las probabilidades de esa etiqueta de pertenecer a una clase u a otra. Para conseguir ésto hay que realizar predicciones por medio del aprendizaje automático, como ya se ha comentado en otras secciones.

Los clasificadores que se han comparado han sido un clasificador de \emph{random forest} y un clasificador \emph{Naive Bayes}. En el primer caso, como podemos indicar cuántos árboles queremos que formen el bosque, hemos realizado pruebas de rendimiento con varios casos, como se puede ver en la siguiente imagen:

\imagen{clasificacionchart}{Comparativa de rendimiento en función del número de árboles}

Como se puede observar, a medida que desciende el número de árboles, más rápidas son las predicciones. No obstante, hay que tener en cuenta que la variación de las mismas sí que se reduce con más árboles. Por tanto, disminuir mucho el número de árboles lleva a predicciones poco estables y en muchos casos, equivocadas. Con 90 árboles, los tiempos de carga son aceptables y los resultados siguen siendo bastante estables, así que fue la cantidad que se escogió como ganadora.

En el caso del clasificador \emph{Naive Bayes}, los tiempos de carga eran mínimos, tardaba entre 3 y 4 segundos en realizar las predicciones. El problema es que este tipo de clasificador no usa \emph{Bagging}, y por tanto, todas las predicciones que haga no van a tener un componente aleatorio y no van a cambiar a no se que cambie el conjunto de entrenamiento. Dicho de otra manera, si el clasificador está realizando predicciones erróneas, no hay manera de saberlo pues no va a variar los resultados con repetidas pruebas.

\subsection{Creación de explicaciones}

En el caso de las explicaciones, la perspectiva es notablemente distinta. Dado que se usa la librería \emph{Lime} para generarlas, como ya se ha explicado en la parte de herramientas, los tiempos de carga no dependen exclusivamente del tipo de clasificación que se esté realizando. 

De hecho, realizando pruebas similares a las del apartado anterior, se ha podido ver que para crear una explicación para una noticia tarda siempre lo mismo, ya sea un clasificador \emph{Random Forest} o un \emph{Naive Bayes}, con márgenes de unos 5 segundos. De hecho, aún probando con cantidades de árboles muy distintas (de 20 a 500), sólo se ha notado efecto en el rendimiento a partir de los 400 árboles, aproximadamente. Usando esto a como una ventaja, se ha podido aumentar mucho el número de árboles en comparación con los que se usan en la predicción, lo cual aumentará la precisión promedio de los resultados.

Desafortunadamente, los tiempos de carga de \emph{Lime} son perceptibles en comparación con el resto de operaciones de la herramienta, y se puede hacer poco a nivel de código para mejorar esta situación.

\section{Experimentación en Jupyter}

Debido a los problemas que se mencionan a continuación, la opción de implementar la aplicación web basada en notebooks de Jupyter tuvo que ser descartada en pleno desarrollo del proyecto.
No obstante, los primeros meses del mismo fueron dedicados casi completamente a la experimentación de funcionalidades en esta plataforma. De hecho, en el repositorio del proyecto se encuentra un grupo de notebooks que no llegaron a formar parte del producto final, pero que a su vez contienen funcionalidad avanzada que tampoco pudo incluirse en la aplicación de Flask. Los ejemplos más interesantes son los siguientes:

\begin{itemize}

\item \emph{Web Scraping} avanzado: Antes de enfocar el web scraping en los RSS de los medios de comunicación. Se realizaron dos prototipos de web scraping convencional en la página web de los periódicos Público y ElDiario.

Cabe destacar que este \emph{web scraping} estaba más perfeccionado que el que hay en el código final, pero al mismo tiempo estaba demasiado personalizado para cada medio en concreto, y no podía replicarse con facilidad para otros portales web. En la versión final, se pueden añadir más medios con menos esfuerzos, a costa de sacrificar calidad en el texto extraído de cada noticia.

\item Clasificación con \emph{Bag of Words}: En algunos de los notebooks hay muestras de las primeras pruebas de clasificación de texto que se hicieron usando Scikit-Learn, en las que además se puede ver como se usan otros elementos como la validación cruzada para crear conjuntos de entrenamiento y de prueba a partir de los mismos elementos iniciales. Como en la aplicación final sabemos a ciencia cierta cuál son las noticias de entrenamiento y cuál sobre la que hay que hacer la predicción, no hace falta usar esta validación cruzada.

\item Lectura más exhaustiva de PDF: También hay un apartado de notebooks dedicado a la lectura de PDF y extracción de su texto, donde se implementaron los primeros acercamientos antes de programar la versión final en la aplicación de Flask. No obstante, esa versión tenía el problema de que los PDF de prueba que se habían recibido tenían diferentes formatos y no se había acordado una estructura común para el contenido, provocando que el texto que se recogía no fuera útil.

\end{itemize}

\section{Intento de publicación de Jupyter Dashboard}

Uno de los objetivos iniciales que se comentaron al comenzar el proyecto consistía en, aparte de conseguir implementar la funcionalidad y la interfaz de usuario en un notebook de Jupyter, poder publicar estos archivos en un formato de aplicación web usando unas librerías llamadas Jupyter Dashboards. La finalidad de estas librerías es la de, teniendo un notebook con la funcionalidad completa, quitar de la pantalla todos los cuadros de código, dejando solo el ``output'' y los elementos interactivos a la vista; ideal para una situación como la que se nos presentaba, en la que la aplicación la iban a usar otros usuarios.

No obstante, los requisitos para poder realizar esta publicación fueron complicándose hasta el punto de tener que descartar esta posibilidad. En lo referente al formato de la salida por pantalla, si que se consiguió configurar el notebook para que tuviera este aspecto de \emph{dashboard}, pero todos los problemas estaban relacionados con la publicación del mismo. Por un lado, era necesario configurar una máquina virtual Docker que ejecutara los comandos necesarios para instalar y configurar todos los elementos que requería este tipo de aplicación. 

Para realizar este proceso, y debido a lo prematuro de estas herramientas (Muchos de los problemas que sufrimos aparecen como issues sin resolver en los respectivos repositorios de GitHub), la opción más viable era seguir el único repositorio de GitHub con guías de despliegue de \emph{dashboards} (\url{ https://github.com/jupyter-incubator/dashboards_setup}) que los propios autores habían facilitado. De estas guías, tres necesitaban de docker para funcionar y la cuarta recurría a CloudFoundry. Mientras que CloudFoundry se descartó porque requería de versiones de pago o versiones de prueba de alguna de las plataformas certificadas que usaban esta tecnología, en el caso de Docker las complicaciones eran de otra naturaleza.

Debido a las condiciones del equipo en el que se realizó el proyecto (Sistema Operativo Windows con Hyper-V), no se podían configurar máquinas docker desde la máquina virtual, porque las máquinas virtuales con Linux que estábamos usando (en este caso, de Oracle VirtualBox) no funcionan con Hyper-V, y las máquinas virtuales de docker necesitaban crearse con Hyper-V activado. Ante la obligación de configurar todo desde el Sistema Operativo anfitrión, un conjunto de incompatibilidades y errores sin aparente solución entre Windows y estas máquinas virtuales, hizo que los recursos necesarios para solventar esta situación fueran demasiado grandes para las limitaciones de tiempo del proyecto.


\section{Omisión de comentarios en el Scrapping}

Una de las funcionalidades que se pretendía incluir en el producto, era la lectura y clasificación por separado de los comentarios de las noticias buscadas. El principal motivo era que los comentarios de una noticia no tenían por qué tener la misma ideología que la noticia, y de hecho muchos de ellos podrían ser de crítica de la propia noticia, aunque se suponga que la mayoría de lectores de un diario en concreto probablemente tenga la misma línea ideológica que el propio medio.


Para realizar este scrapping por separado, se probaron diferentes alternativas, aunque ninguna era muy compatible con las técnicas de \emph{Web Scrapping} que se usaban para leer el resto de la página y, en resumen, provocó la aparición de dos inconvenientes importantes:

\subsection{Uso de Selenium para comentarios ocultos}

En la mayoría de medios en los que se realiza la lectura de noticias, surgió el problema de que al hacer \emph{web scrapping} con \emph{urllib2}, no aparecían los comentarios en el código HTML que llegaba al código del programa. Esto se debe a que en muchos casos los comentarios de los medios se cargan usando herramientas de terceros que dependen de \emph{iframes} y que, al usar un tipo concreto de \emph{request} para coger el código HTML, no se podía alcanzar desde el código Python. 

Por tanto, para atajar este problema, la alternativa más exitosa fue usar Selenium para poder acceder a esos iframe. No obstante, la url con el HTML de los comentarios tenía que cargarse en una pestaña aparte, además de que Selenium abre físicamente el navegador al cliente que esté ejecutando el código, lo cual a su vez generaba otros dos problemas:

\begin{itemize}

\item Según el medio o la herramienta que use el mismo para cargar los comentarios en las noticias, el hecho de que haya que hacer nuevas peticiones con las urls de los comentarios para cada noticia es una operación muy pesada que conllevaba varios minutos de carga en muchos casos, algo totalmente no deseable si la aplicación la va a utilizar algún usuario.

\item Además del tiempo que se tarda en realizar estas operaciones, el usuario además tendría que aguantar que se estuvieran constantemente abriendo pestañas del navegador mientras se realiza la lectura de comentarios, lo cual tampoco es deseable si es un producto que va a usar un tercero.

\end{itemize}
 


\subsection{Falta de comentarios en RSS}

Uno de los principales inconvenientes producidos por el cambio de enfoque de Web Scrapping al RSS de los diarios fue la omisión de lectura de comentarios de las noticias. Debido a las características de este formato, no se incluyen comentarios en las fichas de las noticias en los RSS de los medios. Debido a ello, a pesar de las ventajas que supone este formato en otros aspectos, como facilidad de lectura de noticias y velocidad de la propia lectura, una de las prestaciones que había que sacrificar era la lectura de comentarios.

Incluso en el caso de aquellos medios en los cuáles hay que acceder al enlace que ofrece el RSS para leer el cuerpo de la noticia, si también quisiéramos aprovechar para leer comentarios, se volvería al problema mencionado previamente de falta de opciones sencillas de recogida de los mismos.


Por estos dos motivos, se decidió prescindir de la lectura de comentarios en el producto entregable, a falta de encontrar una alternativa menos intrusiva y con menos costes de tiempo y recursos.
\capitulo{6}{Trabajos relacionados}

El campo de los análisis de texto y de la minería de datos sobre texto ya cuenta con numerosas aportaciones de gran calidad, aunque cabe destacar que en español son bastante más escasos que en el panorama angloparlante.

Aunque no se ha tomado ningún trabajo previo como claro referente, sí que se pueden mencionar varios a los que se ha acudido tanto para obtener información general como para documentarnos acerca de ciertas técnicas relacionadas con las implementadas en nuestra herramienta.

Por ejemplo, la guía de machine learning para reconocer spam que Kevin Markham expuso en la Pycon de 2016 \cite{spamtutorial} tiene un planteamiento similar al de este proyecto, además de servir de tutorial para el uso de Scikit-Learn, una herramienta crucial en la aplicación que se ha desarrollado. Este programa usaba el bag of words en diferentes frases para predecir qué nuevas frases podrían ser consideradas spam o no. Aparte, aportaba datos muy interesantes como las probabilidades de una palabra concreta de aparecer en textos con una clase u otra.

Otro trabajo que se tuvo en cuenta en varios puntos del desarrollo fue éste de Manuel Garrido \cite{tweetsmap}, en el que se pretende hacer un análisis de sentimientos en Español, usando como corpus un gran conjunto de tweets creado por el Taller de Análisis de Sentimiento en Español, como bien explica en el artículo ``Cómo hacer Análisis de Sentimiento en español" \cite{spanishsentiment}. Además, utiliza herramientas como TextBlob, que también fue usada en diferentes puntos de nuestro proyecto.

Por último, en el artículo ``Machine Learning in Automated Text Categorization" \cite{Sebastiani:2002:MLA:505282.505283} de Fabrizio Sebastiani se profundiza bastante en todo lo relacionado con la categorización de textos usando machine learning, describiendo diferentes alternativas para realizarlo.
\capitulo{7}{Conclusiones y Líneas de trabajo futuras}

\section{Conclusiones}

Tras ver toda la funcionalidad recogida en la aplicación final, se puede concluir que el proyecto ha alcanzado los objetivos propuestos. Especialmente, si se tiene en cuenta que los únicos objetivos iniciales claramente delimitados eran los relacionados con la lectura de noticias, y que en lo referente a minería de datos la intención era experimentar y conseguir toda la funcionalidad que se pudiera programar en el tiempo restante del proyecto.

Recopilando todo lo anterior, se ha logrado desarrollar una herramienta en la que un usuario puede gestionar una hemeroteca de noticias que permite la lectura de las mismas en diferentes formatos y su posterior etiquetado.

Además se han añadido varios aspectos de machine learning que permiten hacer predicciones de etiquetado de las noticias alternativas al etiquetado manual.

Finalmente, se han aplicado múltiples conceptos aprendidos a lo largo de la carrera, tanto de programación y diseño como de planificación del trabajo realizado.

No obstante, hay varios puntos en los que hay posibilidad de mejora, o de añadido de nuevas prestaciones que complementen lo existente, que se detallarán en el siguiente punto.

\section{Líneas de trabajo futuras}

Una de los aspectos donde hay más margen de mejora es en el método de predicción de noticias. Aunque el enfoque elegido en la aplicación final ha sido el del bag of words, inicialmente se contempló usar frases o párrafos enteros en lugar de palabras como base para el machine learning. Ese es el motivo de la experimentación con TextBlob y que en algún notebook usara este enfoque. 

Por problemas relacionados con los corpus de noticias y una peor compatibilidad con Lime, se decidió finalmente descartar este enfoque y usar bag of words con Scikit-learn. Aún así, las herramientas y librerías de análisis de sentimientos de párrafos son una aproximación muy interesante y que, usada correctamente, podrían mejorar la calidad de las predicciones.

Otro apartado que queda pendiente es el de la publicación de la aplicación en un servidor web. Como ya se ha comentado previamente, se hicieron varias pruebas cuando se usaba Jupyter como herramienta de desarrollo, pero no se logró una correcta publicación. Por otra parte, con la aplicación programada en Flask no hubo tiempo para investigar como realizar esta tarea de forma correcta; pero se presupone que, por el formato de la aplicación, la futura subida en un servidor no será especialmente complicada de llevar a cabo.

En lo referente al Web Scraping, una tarea pendiente relevante sería encontrar la manera de recoger los comentarios de las noticias evitando los problemas de rendimiento (entre otras cosas) ya mencionados que se sufrieron al usar Selenium como solución. Los comentarios pueden aportar mucho a estos análisis ideológicos de noticias, ya que las opiniones de los lectores de un medio, u otros aspectos como la toxicidad de los comentarios, pueden afectar mucho al conjunto de opiniones relacionado con esas noticias.

Otros aspectos menos importantes pero que también se deberían tener en cuenta son los siguientes:

\begin{itemize}

\item Mejora del aspecto gráfico de la aplicación.

\item Añadir filtros adicionales de búsqueda de noticias en base de datos, no sólo por palabra clave.

\item Conseguir dejar intactos los acentos de los textos de entrada, sorteando los problemas de codificación de Python, para mejorar la calidad de las predicciones.

\item Crear un sistema de usuarios, en el caso de que la aplicación fuera utilizada por más de un usuario, de modo que cada uno tuviera su propia hemeroteca.

\item Permitir que los dataset puedan soportar más de dos clases.

\item Localizar la aplicación a otros idiomas, como el inglés.

\end{itemize}


\bibliographystyle{ieeetr}
\bibliography{bibliografia}

\clearpage

\clearpage
\thispagestyle{empty}
\begin{vplace}
\begin{center}
\includegraphics[width=0.9\textwidth]{logo_COLOR_2L_ABAJO}
\end{center}
\end{vplace}

\end{document}
