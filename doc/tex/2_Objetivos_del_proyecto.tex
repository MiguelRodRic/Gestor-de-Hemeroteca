\capitulo{2}{Objetivos del proyecto}

\section{Objetivos basados en los requisitos funcionales}

Los siguientes objetivos se corresponden con las funcionalidades que debería satisfacer la aplicación:

\begin{itemize}

\item Almacenar noticias de distintas páginas web con información esencial: autor, fecha, titular, texto... etc.

\item Leer noticias que se adjunten en un formato PDF en un directorio concreto, cuyo contenido tenga un formato prefijado para facilitar la lectura.

\item Poder asignar una clase a estas noticias respecto a un dataset, ya sea de forma manual, o por medio de una predicción realizada con minería de datos.

\item La posibilidad de incluir un dataset entero de noticias mediante un archivo XML, con etiquetas ya añadidas, por si el usuario quiere añadir un corpus de entrenamiento directamente.

\item Mostrar una explicación de la predicción que se realice de las noticias, de modo que el usuario sea capaz de observar cuáles han sido los criterios para sugerir una clase u otra.

\item Añadir nuevos datasets de forma manual, y clasificar las noticias existentes para esos nuevos datasets.

\end{itemize}


\section{Objetivos técnicos}

Los siguientes objetivos están relacionados con los aspectos técnicos que la aplicación debe cumplir: 

\begin{itemize}

\item La aplicación debe poder funcionar, al menos en un despliegue local, en el ordenador del usuario.

\item La aplicación en general debe tener un rendimiento óptimo, aunque en ciertos puntos claves los tiempos de carga puedan sufrir aumentos.

\item Al ser una aplicación web, se espera que funcione correctamente en los navegadores web más populares, como pueden ser Mozilla Firefox o Google Chrome.

\item Reducir todo lo posible el proceso de configuración y ejecución de la aplicación al exportarla a nuevos equipos.

\end{itemize}

\section{Objetivos personales}

Los siguientes puntos resumen los objetivos del alumno respecto de este proyecto:

\begin{itemize}

\item Poder trabajar en un proyecto con mayor envergadura de lo habitual siguiendo una metodología de trabajo ágil.

\item Adquirir conocimientos relacionados con el machine learning y minería de datos sobre texto. Especialmente, poder trabajar con ejemplos reales de análisis de textos (etiquetado de noticias).

\item Aprender a manejar bases de datos NoSQL, como MongoDB.

\item Mejorar las capacidades y experiencia de programación con Python y utilizar librerías de terceros muy populares en la minería de datos.

\item Aprender a crear aplicaciones web usando Python como lenguaje back-end, en vez de otras opciones más convencionales como C\# o PHP.

\item Experimentar con herramientas como Jupyter, que tienen una creciente popularidad, ya que son entornos alternativos de programación interesantes y prácticos.


\end{itemize}