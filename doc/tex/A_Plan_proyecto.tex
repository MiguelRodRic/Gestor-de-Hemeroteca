\apendice{Plan de Proyecto Software}


\section{Introducción}

A continuación se va a describir en profundidad cuál ha sido la planificación del proyecto, semana a semana, organizado en base a sprints.

Para tener una visión apropiada de todo el proceso, se ha usado la extensión ZenHub que se puede añadir a GitHub y que permite observar con más detalle todo lo relacionado con la metodología ágil. El repositorio de este proyecto es \url{https://github.com/mrr0088/Gestor-de-Hemeroteca}, aunque ya se incluyó al usuario ``ubutfgm" para que las observaciones necesarias sean realizadas.

Cabe destacar que, especialmente en los primeros sprints, por falta de experiencia con la herramienta ZenHub, los gráficos de progreso no son representativos del progreso real que se llevó a cabo, debido sobre todo a que los issues que los componen no estaban cerrados correctamente.


\section{Planificación temporal}

Como ya se ha mencionado, el desarrollo se dividió en sprints, equivalentes a una semana, excepto en un sprint de dos semanas. Cada comienzo y final de sprint coincide con la reunión semanal del alumno con los tutores.

En nuestro caso, hemos considerado que los story points asignados a cada tarea simbolicen las horas que se tarda en resolver ese issue. Por ejemplo, si una tarea tiene 5 story points se estima que el alumno la realizará en 5 horas.

\textbf{Primer Sprint (2 de Febrero - 10 de Febrero):} 

Este tramo inicial fue dedicado a realizar investigación sobre las herramientas que se usarían para desarrollar el producto final. Se buscó información sobre 3 elementos o partes distintas:

\begin{itemize}

\item Herramienta para desarrollar la GUI: Concretamente, se debatió acerca de si sería más adecuado crear una aplicación web o una aplicación interna. Para tomar una decisión, se tuvieron en cuenta 4 herramientas. Por la parte de aplicación web, las candidatas fueron Jupyter y Flask; y por la parte de aplicación de escritorio, PyQt y Kivy. Tras practicar con ambas herramientas, se tomó la decisión de programar la GUI en un notebook de Jupyter para posteriormente publicarlo en un Dashboard.

\item Base de datos para almacenar la información: Como se debía almacenar numerosa y variada información sobre las noticias elegidas, se necesitaba de una estructura donde almacenar todos esos datos. La decisión más relevante en este apartado era si usar una base de datos relacional o no relacional, o dicho de otra manera, SQL o NoSQL. Las alternativas fueron AlchemySql (relacional), PyLucene (no relacional) y PyMongo (no relacional).

\item Herramienta para escribir la documentación: Aquí sólo se plantearon dos alternativas inicialmente: OpenOffice o Latex. Por cuestiones de formato y de preferencia de los miembros del proyecto, se optó por Latex, aprovechando la plantilla para Trabajos de Fin de Grado que ya había sido creada y que estaba a disposición de los alumnos.

\end{itemize}

\imagen {Issues1}{Issues del primer sprint}

\imagen {Report1}{Progreso de resolución de tareas del primer sprint}

\textbf{Segundo Sprint (10 de Febrero - 16 de Febrero):}

La prioridad de este sprint era finalizar la investigación y búsqueda de herramientas para desarrollar la aplicación, que concluyó con la elección de Pymongo como base de datos, y los notebook de Jupyter como herramienta de interfaz.
Una vez elegidas las diferentes herramientas necesarias para el desarrollo del proyecto, el siguiente paso consistió en crear un pequeño programa que probara la librería PDFMiner y que contuviera algo de funcionalidad básica como contar la frecuencia de una palabra dentro de un documento.
Aprovechando este script, se realizaron pruebas con un snippet llamado fileupload, que permite cargar ficheros en el código usando una GUI clásica en la que puedes navegar por el equipo buscando la ubicación del mencionado fichero. El resultado final era un programa que  permite al usuario buscar un fichero en el equipo y, si este archivo tiene una extensión pdf, se permitirá buscar la frecuencia de las palabras que introduzca el propio usuario.

\imagen {Issues2}{Issues del segundo sprint}

\imagen {Report2}{Progreso de resolución de tareas del segundo sprint}

\textbf{Tercer Sprint (16 de Febrero - 23 de Febrero):}

Habiendo finalizado la elección de todas las herramientas necesarias, y probado el funcionamiento básico para lectura de PDF subiendo archivos a mano, la siguiente prioridad fue crear un primer prototipo de web scrapping.
Para lograr esto, se dividió la funcionalidad en dos etapas:
\begin{itemize}

\item Un primer script que realiza la lectura web de noticias en diferentes medios (para este prototipo, se leían noticias sólo del diario “Público” y, más en concreto, de la sección de Igualdad), almacenando en la base de datos estructuras formadas por el titular, autor, fecha, texto de la noticia, procedencia y un link a la misma.
Para recoger el texto html de las páginas web de los diarios, se usaron las librerías de url2lib y BeautifulSoup; la primera para hacer una llamada a la url indicada y obtener todo (o casi todo, en las próximas etapas explicaremos qué faltaba) el texto HTML de la página de cada noticia, y la segunda para poder manipular todo el HTML recibido previamente de forma más sencilla sin tener que hacer una división manual por etiquetas, simplemente indicamos al contenido de qué etiquetas queremos acceder.
 
\item Un segundo script que, en función de una palabra clave introducida por el propio usuario, consulte en la base de datos y extraiga todas las noticias que contengan esa palabra clave, en un formato de tabla HTML y mostrando la información más importante de la noticia junto con el número de apariciones que hace esa palabra en el texto de las mismas.
Para poder mostrar esto en la salida del notebook, necesitamos las librerías de pandas, ipywidgets y qgrid. Con la librería de pandas podemos organizar la información devuelta en la consulta de la base de datos a un dataframe, que puede ser convertido a una tabla html. Con los widgets, podemos mostrar controles básicos como botones, entradas de texto, listas desplegables, etc… sin necesidad de escribir a mano el código HTML. En esta versión, se usó la librería qgrid para transformar el dataframe en una tabla interactiva, que permitía reordenar el orden de las filas dando prioridad a unas columnas u otras. No obstante, no permitían formatear las celdas por separado, lo cual nos hizo descartarlo para posteriores versiones.

\end{itemize}

\imagen {Issues3}{Issues del tercer sprint}

\imagen {Report3}{Progreso de resolución de tareas del tercer sprint}

\textbf{Cuarto Sprint (23 de Febrero - 2 de Marzo):}

Habiendo realizado una primera aproximación tanto a la lectura de PDF como de web, este sprint se centró en mejorar tres apartados:

\begin{itemize}

\item En la parte de PDF, se mejoró el script, aunque a costa de retirar temporalmente la librería que permitía seleccionar un fichero de forma manual, para que recorriera y almacenara todos los archivos PDF, tanto del directorio que se pasaba como argumento como de todos los que colgaban de él. Una vez recorridos todos, se guardan las noticias en la base de datos, de forma similar a como se hacía con las noticias en la web pero cambiando el atributo de la procedencia.

\item En la parte de web scrapping, se actualizaron dos aspectos importantes:

\begin{itemize}

\item Por una parte, usando Selenium además de BeautifulSoup se consiguió obtener los comentarios de las noticias, que en el caso de “Público”, se generaban con una API de terceros que impedía acceder a su HTML de forma normal como se hacía con el resto de la noticia. El método era algo pesado, ya que selenium necesita abrir en el ordenador que se esté ejecutando las pestañas a las que accede, pero como ya se ha mencionado, fue la alternativa que se encontró más idónea para poder acceder al HTML de los comentarios de las noticias.

\item Por otro lado, se cambió el formato de la tabla, dejando de usar qgrid para el DataFrame y mostrándola en un HTML normal, al que luego se aplicaba estilo manualmente usando un script de Javascript. Ésto es posible ya que uno de los widgets de Ipython/Jupyter permite introducir código HTML de forma manual, y es la vía que también utilizamos para introducir código JavaScript dentro del output del notebook. Con este cambio, tenemos flexibilidad total para manipular las celdas de la tabla para, por ejemplo, formatear como hiperenlaces las url de las noticias que se muestran en la tabla.


\end{itemize}

\end{itemize}

\imagen {Issues4}{Issues del cuarto sprint}

\imagen {Report4}{Progreso de resolución de tareas del cuarto sprint}

\textbf{Quinto Sprint (2 de Marzo - 10 de Marzo):}

En esta etapa se decidió que en el muestreo de resultados, sería interesante incluir gráficos que acompañaran al resto de la información, para poder mostrar la mayor cantidad de datos posibles de diversas formas. Por tanto, una de las tareas fue elegir una librería de gráficos adecuada a nuestro contexto. 
Las candidatas iniciales fueron Bokeh y SeaBorn. Se probó Bokeh y, debido a unos problemas relacionados con la generación de archivos HTML, de decidió descartar. Por falta de horas durante esa semana, se trasladaron las pruebas con SeaBorn al siguiente sprint.
La otra gran tarea a la que se dedicó tiempo durante este sprint fue a aumentar los medios a los que se realizaba web scrapping, creando un script para recorrer y almacenar las noticias de ElDiario. En este caso, los comentarios no se generaban a partir de una API de terceros, pero debido a la complejidad del código HTML obtenido, y visto necesario el uso de Selenium una vez más, se decidió separar la obtención de comentarios y aplazar esa parte al siguiente sprint.

\imagen {Issues5}{Issues del quinto sprint}

\imagen {Report5}{Progreso de resolución de tareas del quinto sprint}

\textbf{Sexto Sprint (10 de Marzo - 16 de Marzo):}

La mayor parte de esta semana se dedicó a terminar tareas de la semana anterior que habían sido subestimadas en horas, y separadas en tareas entre este sprint y el anterior.
En el apartado de los gráficos, se realizaron pruebas con la otra librería candidata, Seaborn, y por su comodidad y sencillez para mostrar gráficos simples (Como gráficos de barras para evaluar las menciones a una palabra por autor o día de la semana), se decidió que por el momento sería la alternativa elegida para acompañar en la interfaz.
Respecto al web scrapping de ElDiario, se implementó el código necesario junto con el uso de Selenium adecuado para poder obtener los comentarios de las noticias de ese medio, que era la funcionalidad que no dio tiempo a terminar en la anterior semana.
Aparte de estas tareas, se dedicó cierta cantidad de tiempo a investigar formas de publicar esta aplicación web en algún servidor. Para comenzar, se realizó cierta investigación sobre Docker, además de ver vídeos de la PyCon de Londres 2016, donde se trataba el tema de subir notebook a un servidor y poder visualizarlos como un dashboard. Por problemas de límite de tiempo, no se pudo avanzar mucho más este aspecto del proyecto.

\imagen {Issues6}{Issues del sexto sprint}

\imagen {Report6}{Progreso de resolución de tareas del sexto sprint}

\textbf{Séptimo Sprint (16 de Marzo - 23 de Marzo):}

En este sprint se dedicó parte del tiempo a solucionar un problema recurrente en el que se mostraban sucesivamente todas las tablas de resultados de todas las palabras buscadas, provocando que la interfaz resultante se llenara de tablas innecesarias. De la misma manera, con los gráficos de Seaborn, que usa MatplotLib para funcionar, sucedía algo similar, provocando que todos los gráficos generados se fueran acumulando en una misma zona de dibujado que comparten por requisitos de la librería.
Para solucionar esto hicieron falta dos medidas distintas. Para la parte de las tablas html, el problema se solucionó desde el código de Python, cambiando el planteamiento de la función que se ejecutaba al pulsar en el botón de búsqueda, y haciendo globales algunas variables necesarias para la generación de estas tablas. Para la parte de los gráficos, como este problema se estaba produciendo por limitaciones de la librería, hubo que buscar una solución más “rudimentaria” y controlar, por medio de una función JavaScript (introducida en un widget HTML), la visualización de la tabla requerida y la no visualización del resto de tablas obsoletas.
El resto del tiempo de esa semana se dedicó a la investigación de la herramienta TextBlob, una opción muy interesante para realizar minería de datos de texto. En términos generales, es una herramienta que, a partir de los textos recibidos, puede traducirles, detectar idiomas, corregir palabras, dividir párrafos en frases, y frases en palabras, devolver etiquetas de las frases o palabras… etc. Aparte de toda esta funcionalidad, muy útil para casos como el de este proyecto, en el que se manejan textos, tiene sus propios clasificadores de datos. 
Para usar esta herramienta, se necesita descargar el NLTK (Natural Language ToolKit), que contiene todos los conjuntos de datos necesarios para poder llevar a cabo estas manipulaciones de texto y clasificaciones de los mismos. 
Debido a problemas con la descarga y tamaño de los archivos, se cerró la tarea en ese punto, y se retomó en el siguiente sprint.
Al término de esta semana, se realizó una reunión con el cliente para enseñar los progresos alcanzados hasta ese punto y tomar decisiones acerca de qué pasos tomar a continuación. 

\imagen {Issues7}{Issues del séptimo sprint}

\imagen {Report7}{Progreso de resolución de tareas del séptimo sprint}

\textbf{Octavo Sprint (23 de Marzo - 30 de Marzo):}

La tarea más importante realizada esa semana fue la de hacer otra aproximación al web scrapping basada en obtener los datos a partir de los RSS de los medios en los que, hasta el momento, se estaban leyendo desde las páginas web convencionales. El resultado fue un pequeño programa que te permitía seleccionar un medio de una lista de diarios y una palabra clave y, en tiempo real, realizar el web scrapping del RSS de ese medio y almacenarlo en la base de datos a la vez que se mostraba en una tabla los resultados obtenidos. Para ello se usó la librería feedparser.
Este acercamiento era mucho más rápido que el web scrapping convencional, pero a cambio sacrifica la posibilidad de poder acceder a los comentarios de las noticias, y los casos de algunos medios, no todo el texto de la noticia es accesible a través del RSS.
Como ha sido mencionado en el resumen del sprint anterior, debido a la falta de capacidad en la máquina virtual en la que se estaba trabajando, no se podía descargar todo el conjunto de datos de NLTK de forma correcta, lo cual se intentó solucionar inicialmente clonando la máquina virtual a una de almacenamiento dinámico y editando la capacidad de almacenamiento de las particiones de la misma. 
Esta solución no tuvo resultados positivos, por tanto, se reinstaló desde el principio una nueva máquina virtual que evitara estos problemas y que además corrigiera la instalación de algunas librerías que habían producido problemas previamente.
Por último, se probó una nueva librería de gráficos que sustituyera a Seaborn, denominada Bqplot, la cual se asemejaba en funcionalidad a Bokeh, pero evitando esa necesidad de archivos HTML que nos habían llevado a descartar Bokeh en un primer momento. Las pruebas fueron muy positivas en gráficos de barras y lineales.

\imagen {Issues8}{Issues del octavo sprint}

\imagen {Report8}{Progreso de resolución de tareas del octavo sprint}

\textbf{Noveno Sprint (31 de Marzo - 13 de Abril):}

La duración de este sprint fue el doble de lo normal (dos semanas en vez de una) debido a que coincidió con las vacaciones de Semana Santa. Por tanto, el día que debía haberse realizado la reunión habitual no era lectivo, y hubo que posponerlo. 
Por otro lado, como había suficientes tareas/horas de trabajo como para dividirlo en más de un sprint, se dio por finalizado el sprint a los 14 días aunque la siguiente reunión fuera después de 21.
En lo relacionado a tareas relacionadas con sprints anteriores, se consiguió solucionar el problema por el cual no se aplicaban los scripts de JavaScript, aunque para ello hubo que buscar una forma alternativa de implementarles.
Además, se consiguió finalizar con éxito la instalación y pruebas con TextBlob, llevando a cabo la instalación del NLTK de una manera alternativa que ocupaba mucho menos espacio en memoria. Una vez instalado y probado, se hizo un sencillo prototipo con un conjunto de datos de prueba minúsculo de diferenciador de frases machistas y feministas, tanto con TextBlob como con SciKitLearn, aprovechando código de scripts ya existentes.
También se mejoraron las operaciones de almacenamiento en base de datos, evitando introducir noticias duplicadas, para que el usuario final no se tenga que preocupar de filtrar las que estén duplicadas de las que no.
En lo respectivo a la publicación del notebook como aplicación web, se realizaron instalaciones de varias herramientas de Jupyter Dashboards y de Docker, pero debido a impedimentos de configuración de la BIOS y del anfitrión, no se puedo avanzar más en la máquina virtual, y se decidió repetir el proceso en el anfitrión en el siguiente sprint.
Por último, se añadió el diario “El Mundo” a los disponibles en el web scrapping por RSS, para ir aumentando el rango ideológico de las alternativas disponibles.

\imagen {Issues9}{Issues del noveno sprint}

\imagen {Report9}{Progreso de resolución de tareas del noveno sprint}

\textbf{Décimo Sprint (14 de Abril - 21 de Abril):}

En este periodo se continuó la investigación y las pruebas de despliegue de aplicación de los notebook en una máquina de Docker, aunque en esta ocasión se realizaron en el Sistema Operativo anfitrión (Windows) en vez de en la máquina virtual, para comprobar si al cambiar el entorno conseguíamos solucionar los problemas técnicos sufridos durante la configuración de Docker.

Se realizaron además las primeras pruebas de clasificación de textos con un dataset real, compuesto por un grupo de noticias de diferentes medios de comunicación tratando el debate de los vientres de alquiler, etiquetadas a partes iguales como "A Favor" y "En Contra" del tema que hablaban. Las pruebas fueron realizadas tanto en TextBlob, dividiéndolas en sentencias, como en SciKit-Learn, dividiéndolas en palabras.

También se dedicó parte de la semana a la documentación del trabajo realizado hasta la fecha.

\imagen {Issues10}{Issues del décimo sprint}

\imagen {Report10}{Progreso de resolución de tareas del décimo sprint}

\textbf{Undécimo Sprint (22 de Abril - 27 de Abril):}

Tras los fracasos en la configuración sufridos con Docker, se investigó la alternativa de Cloud Foundry, plataforma open source basada en contenedores que da soporte a la subida de aplicaciones a la nube.

En lo respectivo a la clasificación del Dataset real, se continuaron las pruebas, en este caso probando distintos permutadores (Leave One Out, Leave P Out y ShuffleSplit), para hacer validación cruzada y separar las noticias que teníamos en conjuntos de entrenamiento y conjunto de prueba. 

Además, se investigó la librería de Lime, que permite crear explicaciones detalladas de las predicciones, y se realizaron los primeros experimentos con la misma.

\imagen {Issues11}{Issues del undécimo sprint}

\imagen {Report11}{Progreso de resolución de tareas del undécimo sprint}

\textbf{Duodécimo Sprint (28 de Abril - 4 de Mayo):}

En este punto, se decidió dar más prioridad a crear una aplicación con funcionalidad completa que a seguir invirtiendo más tiempo en intentar configurar un contenedor para poder subir los notebooks de Jupyter con formato de aplicación web. 

Por ello, se comenzó el desarrolló de una aplicación con Flask que reuniera toda la funcionalidad implementada hasta el momento, que estaba repartida en varios archivos sin relación entre ellos, con objetivo de desplegarla en modo local en lugar de subirla a un servidor. 

Se creó un esqueleto con la funcionalidad más básica de lectura de PDF, Web Scraping y persistencia en una base de datos NoSQL, sobre la que construir en las siguientes semanas.

\imagen {Issues12}{Issues del duodécimo sprint}

\imagen {Report12}{Progreso de resolución de tareas del duodécimo sprint}

\textbf{Decimotercer Sprint (5 de Mayo - 11 de Mayo):}

Durante esta semana se continuó la construcción de la aplicación en Flask, como la implementación de las inserciones desde la web a la base de datos, el traslado del código de clasificación de texto de Jupyter a Flask y una mejora en la lectura de PDF.

Aparte de esto, se dedicó un tiempo a solucionar problemas más pequeños como correciones en el uso de rutas literales, sustituir el uso de librerías obsoletas por sus alternativas vigentes, la inclusión de un fichero .txt con las librerías necesarias para ejecutar la aplicación, y completar la experimentación de Lime con los ejemplos de la documentación.

\imagen {Issues13}{Issues del decimotercer sprint}

\imagen {Report13}{Progreso de resolución de tareas del decimotercer sprint}

\textbf{Decimocuarto Sprint (12 de Mayo - 18 de Mayo):}

Este sprint se centró en avanzar todo lo posible la funcionalidad relacionada con la clasificación de texto. Concretamente, se consiguió desde la librería de NLTK una lista de palabras irrelevantes en español a tener en uenta para evitar ruido en los entrenamientos y predicciones. 

En lo respectivo a la aplicación web, también se centraron esfuerzos en la vista de resultados de noticias y las posibilidades de etiquetado de las mismas, permitiendo que las noticias tuvieran varios datasets, que se pudiera añadir un dataset entero mediante un documento XML y se hizo una primera aproximación a un etiquetado manual, pero que sería sustituido por una alternativa más coherente en la siguiente semana.

\imagen {Issues14}{Issues del decimocuarto sprint}

\imagen {Report14}{Progreso de resolución de tareas del decimocuarto sprint}

\textbf{Decimoquinto Sprint (19 de Mayo - 25 de Mayo):}

En esta semana se completó casi toda la funcionalidad restante en la tabla de resultados de noticias, incluyendo una barra de progreso que indicaba las probabilidades que tenía una noticia de pertenecer a una clase o a otra, la posibilidad de que el usuario realice un etiquetado manual a cada noticia, y la posibilidad de ver la explicación de la predicción en una pestaña aparte, acompañada de un gráfico para comprender mejor los resultados.

También se dedicó parte del tiempo a mejorar la coherencia de urls para separar distintas funcionalidades en distintas pestañas, para evitar sobrecargar de complejidad las peticiones de los formularios, sobre todo en las pestañas de lectura de PDF y XML.

\imagen {Issues15}{Issues del decimoquinto sprint}

\imagen {Report15}{Progreso de resolución de tareas del decimoquinto sprint}

\textbf{Decimosexto Sprint (26 de Mayo - 2 de Junio):}

Teniendo gran parte de la funcionalidad implementada, en este Sprint se centró en mejorar el estilo gráfico de la interfaz de la aplicación. Para ello, se usó Bootstrap, primero experimentando con la versión para Flask y posteriormente se decidió permanecer con el modo de uso que tendría en un proyecto web normal, importando los archivos CSS y JS correspondientes. Aprovechando que había que utilizar JQuery para poder aplicar Bootstrap, se refactorizó el código existente de JavaScript en JQuery, reduciendo en un gran tamaño el mismo.

Además del estilo, también se actualizó el modo de lectura de los PDF, creando una estructura dentro de los archivos simple que facilitara la posterior lectura e inserción en base de datos, obligando a todos los documentos PDF que sigan la misma estructura para que puedan ser leídos. 

También se modificó la vista de resultados de la explicación de Lime, aprovechando su característica de poder mostrar una vista de resultados propia exportada como HTML, sin tener que recurrir a librerías de gráficos de terceros.

\imagen {Issues16}{Issues del decimosexto sprint}

\imagen {Report16}{Progreso de resolución de tareas del decimosexto sprint}

\textbf{Decimoséptimo Sprint (2 de Junio - 9 de Junio):}

A estas alturas ya se estaban realizando las últimas mejoras en la funcionalidad, como una actualización de la lectura de los RSS de las páginas web, de los XML y de los PDF. Aparte de ésto, se terminaron los capítulos de la memoria dedicado a los conceptos teóricos y objetivos del proyecto.

\imagen {Issues17}{Issues del decimoseptimo sprint}

\imagen {Report17}{Progreso de resolución de tareas del decimoseptimo sprint}

\textbf{Decimoctavo Sprint (9 de Junio - 15 de Junio):}

En lo relacionado con la documentación, se completaron los capítulos de aspectos relevantes, herramientas utilizadas y las conclusiones y líneas futuras del proyecto.

A lo largo de esta semana también se finalizó la funcionalidad, añadiendo una pestaña para ver gráficos de las noticias almacenadas, además de arreglar los problemas que surgieron al añadir la posibilidad de añadir datasets adicionales. Por último, se implementó la paginación en la lista de resultados, para partir las noticias en pequeños grupos y evitar tiempos de carga demasiado largos.

\imagen {Issues18}{Issues del decimoctavo sprint}

\imagen {Report18}{Progreso de resolución de tareas del decimoctavo sprint}

\textbf{Decimonoveno Sprint (16 de Junio - 22 de Junio):}

Habiendo finalizado la funcionalidad, los objetivos de esta semana se enfocaron en realizar la documentación de los anexos y terminar la de la memoria, a la vez que se realizaban pruebas sobre el código y se realizo refactorización sobre el mismo.


\section{Estudio de viabilidad}

\subsection{Viabilidad económica}

Para poder hacer un uso correcto de la aplicación, no hay que hacer ninguna inversión económica en software de ningún tipo, ya que todas las librerías que se han usado son open-source.

Más allá de lo comprendido por la propia herramienta, el único coste imprescindible sería el del propio equipo en el que se instale y configure. Aunque algunas funcionalidades tengan tiempos de carga más notables que otras, no debería ser impedimento para que la aplicación funcionara correctamente en equipos de diferentes prestaciones. Respecto al Sistema Operativo, como la aplicación se ejecuta en sistemas Linux, no es necesario ni realizar la compra de un Sistema Operativo de pago.

Cabe destacar que en el contexto de este proyecto la aplicación no está publicada en un servidor web, pero que en un futuro podría llevarse a cabo esta operación. De ser así, habría que evaluar diferentes alternativas de host compatibles y comprobar si alguna de estas alternativas son de pago o gratuitas.

\subsection{Viabilidad legal}


