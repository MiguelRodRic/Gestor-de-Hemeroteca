\capitulo{1}{Introducción}

Los estudios sociológicos tienen un gran papel a la hora de comprender y describir comportamientos y fenómenos colectivos de la sociedad \cite{wiki:estudiosociologico}. Además de ésto, también sirven para que determinadas organizaciones den visibilidad a problemas que podrían no ser obvios para una gran parte de la población.

No obstante, las encuestas y métodos similares de recogida de datos no son las únicas formas de recabar información interesante respecto de la sociedad. Por ejemplo, realizando un análisis de otros campos como la prensa escrita, redes sociales o programas de televisión se puede encontrar patrones e intenciones donde se critican o defienden diferentes asuntos de actualidad de forma sistemática, especialmente cuando estas organizaciones tienen marcadas líneas ideológicas.

Un debate especialmente popular en los últimos meses es el tratamiento de la violencia de género por parte de los medios de comunicación, donde se pueden observan claras tendencias a denunciarlo o a enmascararlo, dependiendo de la fuente de las noticias \cite{RICD2653}.

Por este motivo, se decidió realizar este proyecto en colaboración con María Isabel Menéndez Menéndez, doctora e investigadora en la Universidad de Burgos, cuyo área de estudio se centra en el análisis de la comunicación desde la perspectiva de género. 

La intención es crear una aplicación en la que pueda crear una hemeroteca donde realizar estudios sociológicos a partir de patrones encontrados en las noticias almacenadas. Aunque inicialmente se limita a debates como el machismo y el vientre de alquiler, es posible crear en la herramienta más temas de debate para análisis sociológico y reutilizar las noticias para estas nuevas perspectivas.

