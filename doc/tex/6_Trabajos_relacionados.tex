\capitulo{6}{Trabajos relacionados}

El campo de los análisis de texto y de la minería de datos sobre texto ya cuenta con numerosas aportaciones de gran calidad, aunque cabe destacar que en español son bastante más escasos que en el panorama angloparlante.

Aunque no se ha tomado ningún trabajo previo como claro referente, sí que se pueden mencionar varios a los que se ha acudido tanto para obtener información general como para documentarnos acerca de ciertas técnicas relacionadas con las implementadas en nuestra herramienta.

Por ejemplo, la guía de machine learning para reconocer spam que Kevin Markham expuso en la Pycon de 2016 \cite{spamtutorial} tiene un planteamiento similar al de este proyecto, además de servir de tutorial para el uso de Scikit-Learn, una herramienta crucial en la aplicación que se ha desarrollado. Este programa usaba el bag of words en diferentes frases para predecir qué nuevas frases podrían ser consideradas spam o no. Aparte, aportaba datos muy interesantes como las probabilidades de una palabra concreta de aparecer en textos con una clase u otra.

Otro trabajo que se tuvo en cuenta en varios puntos del desarrollo fue éste de Manuel Garrido \cite{tweetsmap}, en el que se pretende hacer un análisis de sentimientos en Español, usando como corpus un gran conjunto de tweets creado por el Taller de Análisis de Sentimiento en Español, como bien explica en el artículo ``Cómo hacer Análisis de Sentimiento en español" \cite{spanishsentiment}. Además, utiliza herramientas como TextBlob, que también fue usada en diferentes puntos de nuestro proyecto.

Por último, en el artículo ``Machine Learning in Automated Text Categorization" \cite{Sebastiani:2002:MLA:505282.505283} de Fabrizio Sebastiani se profundiza bastante en todo lo relacionado con la categorización de textos usando machine learning, describiendo diferentes alternativas para realizarlo.